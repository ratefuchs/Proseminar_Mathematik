\documentclass[a4paper,11pt]{report}
\usepackage{graphicx}
\usepackage[utf8]{inputenc}
\usepackage[T1]{fontenc}
\usepackage[ngerman]{babel}
\usepackage{bibgerm}
\usepackage{amsmath,amssymb,amsthm}
\usepackage{color}
\usepackage{enumerate}
\usepackage{tabularx}
\usepackage{subfig}
\usepackage{fancyhdr}
\usepackage{upgreek}
\usepackage{graphicx}
\usepackage{framed}
\usepackage{lmodern}
\usepackage{geometry}
\geometry{a4paper,left=20mm,right=30mm, top=3cm, bottom=2cm} 
\usepackage{marginnote}
\usepackage[pdftex,pdfpagelabels,colorlinks,backref,pagebackref,plainpages=false]{hyperref}
\setlength{\headwidth}{15cm}
\setlength{\textwidth}{15cm}

\usepackage{shadethm}
% == Set the heading style ===================================================
\setlength{\headheight}{14pt}
%\pagestyle{fancyplain}
\renewcommand{\chaptermark}[1]{\markboth{#1}{}}
\renewcommand{\sectionmark}[1]{\markright{\thesection\ #1}}
\lhead[\fancyplain{}{\thepage}]{\fancyplain{}{\rightmark}}
\rhead[\fancyplain{}{\leftmark}]{\fancyplain{}{\thepage}}
\cfoot{}
\renewcommand{\headrulewidth}{0.4pt}
% ============================================================================

% == Set correct values for fitting floats ===================================
\tolerance=2000
\emergencystretch=10pt

\setcounter{topnumber}{3}
\setcounter{totalnumber}{5}
\setcounter{bottomnumber}{2}

% To make those darn floats fit where they should
\setcounter{totalnumber}{9}
\setcounter{topnumber}{9}
\setcounter{bottomnumber}{9}
\renewcommand{\textfraction}{0.00}
\renewcommand{\topfraction}{1.0}
\renewcommand{\bottomfraction}{1.0}
% ============================================================================

% == German definitions for theorems etc. ==================================== 
%\newtheorem{theorem}{Satz}[chapter]
\newtheorem{lemma}{Lemma}[chapter]
\newtheorem{proposition}{Proposition}[chapter]
%\newtheorem{corollary}{Korollar}[chapter]
\newtheorem{observation}{Beobachtung}[chapter]
\newtheorem{fact}{Fakt}[chapter]
%\theoremstyle{remark} 
\theoremstyle{definition} 
\newtheorem{remark}{Bemerkung}[chapter]
\newtheorem{example}{Beispiel}[chapter]
% ============================================================================

% == Abkürzungen für die reellen, natürlichen, ganzen,... Zahlen =============
\newcommand{\R}{{\ensuremath{\mathbb{R}}}}
\newcommand{\N}{{\ensuremath{\mathbb{N}}}}
\newcommand{\Z}{{\ensuremath{\mathbb{Z}}}}
\newcommand{\C}{{\ensuremath{\mathbb{C}}}}
\newcommand{\Q}{{\ensuremath{\mathbb{Q}}}}
\newcommand{\F}{{\ensuremath{\mathbb{F}}}}
\newcommand{\Prim}{{\ensuremath{\mathbb{P}}}}
% ============================================================================

\newcommand{\RM}[1]{\MakeUppercase{\romannumeral #1{}}} 

% == Makros für Autorenname und -adresse =====================================
\newcommand{\myaddress}[6]{%
  \parbox{\textwidth}{\textbf{\large #1}\\
    #2\\ #3\\ #4\\ 
    \ifthenelse{\equal{#5}{}}{}{Email: \href{mailto:#5}{\texttt{#5}}\\}
    \ifthenelse{\equal{#6}{}}{}{WWW: \href{#6}{\path|#6|}\\}
  } 
}

\newcommand{\myauthor}[1]{%
  \addtocontents{toc}{\protect\hspace{3.35ex}%
  \textsl{#1}\par}\vspace{-4ex}\quad\hfill\textsl{\Large #1}\vspace{8ex}}

\newcommand{\myname}[1]{\Large #1}

%%%%%%%%%%%%%%%%%%%%%%%%%%%%%%%%%%%%%%%%%%%%%%%%%%
% Tragen Sie in der folg. Zeile Ihren Namen ein: %
%%%%%%%%%%%%%%%%%%%%%%%%%%%%%%%%%%%%%%%%%%%%%%%%%%

\newcommand{\OO}{{\ensuremath{\mathcal{O}}}}
\newcommand{\const}{\ensuremath{\equiv}}

%\renewcommand{\thechapter}{\Roman{chapter}}
\renewcommand{\thesection}{\arabic{section}}


\newenvironment{Kasten}[1]
{
\hspace{0.05\linewidth}
\begin{center}
\begin{minipage}{0.95\linewidth}
\setlength{\fboxsep}{10pt}
%\setlength{\fboxsep}{18pt}
%\definecolor{shadecolor}{gray}{0.9}
\definecolor{shadecolor}{rgb}{0.9,1,1}
\definecolor{framecolor}{gray}{0}
%\def\FrameCommand{\fcolorbox{framecolor}{shadecolor}}
%\MakeFramed {\FrameRestore}
\subsection*{#1}
%\begin{itshape}
}
{
%\end{itshape}
%\endMakeFramed
\end{minipage}
\end{center}
%\vspace{1em}
}


\newenvironment{bsp}[1]
{
\setlength{\fboxsep}{10pt}
\subsection*{Beispiel: #1}
\begin{upshape}
}
{
\end{upshape}
}

\newshadetheorem{theorem}{Satz}[chapter]
\newshadetheorem{corollary}{Korollar}[chapter]
\newshadetheorem{definitions}{Definition}[chapter]

\newenvironment{definition}[1]{
	\begin{definitions}
	\marginnote{\emph{#1}}
}{\end{definitions}}


\begin{document}
\pagenumbering{alph}
\begin{titlepage}
	\begin{center}	
		\LARGE \textbf{$n$-dimensionale Polarkoordinaten \\[5ex]
			{\Large Proseminar Mathematik - Themen zur Analysis \\[5ex] 
    		Wintersemester 2012/2013}\\[5ex]}
	\end{center}
	\begin{center}
		Simon Bischof; 29.11.2012 %TODO: exaktes Datum
	\end{center}
\end{titlepage}
%\maketitle
\clearpage\pagenumbering{Roman}
\setcounter{tocdepth}{1}
\tableofcontents

\shorthandoff{"}
\clearpage\pagenumbering{arabic}
\chapter{Motivation}
Wenn mehrdimensionale Integrale in der Vorlesung eingeführt werden, d.h.
$$\int\limits_M f(x)dx \qquad (x\in\R^n,M\subseteq \R^n, f:\R^n\to\R),$$ %TODO:globale Nummer
ist eine der ersten Funktionen, die integriert wird, $f\const 1$. Damit wird der Inhalt (je nach Dimension Fläche, Volumen,$\cdots$) berechnet.\\
Für einfache Flächen wie den $n$-dimensionalen Würfel %TODO:Bilder Würfel & Kugel für n=1..3
ist mithilfe des Satzes von Fubini noch leicht machbar (man bekommt den Inhalt $1$). Für eine weitere regelmäßige Fläche, die $n$-dimensionale Kugel $D^n:=\{(x_1,\cdots,x_n)\in\R^n|x_1^2+\cdots+x\_n^2\leq 1\}$, ist das ganze ungleich komplizierter. Fubini liefert
$$.$$%TODO:Formel
Nach Einführung der Polarkoordinaten lässt sich das Integral mithilfe der Substitutionsregel
$$$$%TODO:Formel+glob. Nr.
viel einfacher berechnen.
\chapter{Vorgehensweise}
Die Polarkoordinaten werden als Funktion $P_n$ vom \glqq Polarkoordinatenraum\grqq{} in den kartesischen Raum induktiv definiert. Dabei dienen die bekannten Polarkoordinaten im Fall $n=2$ als Induktionsanfang. Außerdem werde ich immer noch den Fall $n=3$ explizit angeben, bei dem sich die bekannten Kugelkoordinaten ergeben.\\
%TODO: Anschauung mit Drehung, Bilder,...???
Darauf werden einige Eigenschaften der Funktion $P_n$ bewiesen, insbesondere wird auch die Funktionaldeterminante berechnet, die für die Substitutionsregel %TODO:Referenz
wichtig ist.
%TODO: Anmerkung bzgl Q_n mit r\in\R, phi_1..n-1\in[-pi/2,pi/2]
\chapter{$P_n$: Definition und Eigenschaften}
\section{Definition}
\section{$||P_n(r,\phi_1,\cdots,\phi_{n-1})||$}
\section{Funktionalmatrix und -determinante}
\chapter{Integralberechnungen}
\section{Inhalt der $n$-dimensionalen Kugel}
\section{Sonstiges}%TODO:evtl. auch rausnehmen
\chapter{Verallgemeinerungen}
%TODO:andeuten, dass nur kurz
\section{\glqq Ellipsoidkoordinaten\grqq}%Kommentar: waren bei einer Aufgabe auf einem HM2-Übungsblatt tatsächlich hilfreich...
\section{verallgemeinerte Zylinderkoordinaten}
\end{document}